\documentclass[12pt,a4paper]{article}
\usepackage{fontspec}    % 字体设置包
\usepackage{xeCJK}       % 中文处理包
\usepackage{graphicx}    % 插入图片
\usepackage{amsmath}     % 数学公式
\usepackage{amssymb}     % 数学符号
\usepackage{geometry}    % 页面布局
\usepackage{titlesec}    % 标题格式
\usepackage{enumitem}    % 列表设置
\usepackage{lipsum}      % 生成示例文本

% 页面设置
\geometry{margin=1.5in}  % 页边距

% 字体设置
\setmainfont{Times New Roman}          % 英文主字体
\setCJKmainfont[BoldFont=SimHei, ItalicFont=KaiTi]{SimSun}  % 中文主字体
\setCJKsansfont{SimHei}                % 中文无衬线字体
\setCJKmonofont{SimSun}                % 中文等宽字体

% 标题格式设置
\titleformat{\section}{\centering\Large\bfseries}{\thesection}{1em}{}
\titleformat{\subsection}{\large\bfseries}{\thesubsection}{1em}{}

% 标题信息
\title{XeLaTeX 中文示例文档}
\author{作者名称}
\date{\today}  % 使用当前日期

\begin{document}

\maketitle  % 生成标题

\begin{abstract}

\end{abstract}
\newpage
\makecontent

\newpage

\section{引言}

\section{材料与方法}
\subsection{材料与试剂}

\subsection{微生物培养基配制}

\subsection{微生物培养环境}

\subsection{实验方法}


\section{结果与讨论}
\subsection{结果}
\subsubsection{单一材料比较数据}
\paragraph{铁片}

\paragraph{陶瓷片}

\subsubsection{同种液体比较数据}
\paragraph{无菌水}

\paragraph{沁葡水}

\paragraph{乌龙茶}

\subsubsection{综合比较}

\subsection{分析与讨论}
\subsubsection{最佳承装液体}

\subsubsection{最佳承装容器}

\subsubsection{原因分析}

\section{结论与创新点}
\subsection{结论}
综上所述,我们根据实验数据与严谨的处理、分析过程后得出如下结论:
\begin{itemize}
    \item 结论1
    \item 结论2
    \item 结论3
\end{itemize}

\subsection{创新点}
在我们的研究中,我们实现了如下创新:
\begin{itemize}
    \item 创新点1
    \item 创新点2
    \item 创新点3
\end{itemize}

\section{问题与反思}\label{sec:problem}
\subsection{现有问题}
在本次研究中,出现了如下问题,可能会导致实验数据偏差从而得出错误的结论,包括但不限于:
\begin{enumerate}
    \item 问题1
    \item 问题2
\end{enumerate}

\subsection{反思与改进方案}
针对如上问题\ref{sec:problem},我们提出了如下解决方案:
%IMPORTANT!:这是有序列表,有编号的,每个点应当与问题一一匹配
\begin{enumerate}
    \item 解决方案1
    \item 解决方案2
\end{enumerate}

\subsection{未来展望}
针对以上问题及解决方案,我们希望未来能够进一步跟进研究,得出更多可靠、更完备的结论,包括但不限于:
\begin{itemize}
    \item 展望1
    \item 展望2
\end{itemize}

\section{收获}
%我不知道怎么写

% 致谢页通常单独成页,不编号
\clearpage
\thispagestyle{empty} 
\section*{致谢} % 带星号的section表示不编号

本论文的顺利完成,离不开众多师长、同学和亲友的鼎力支持与帮助。在此,我们谨向他们致以最诚挚的谢意。

首先,我要衷心感谢我的导师张阳老师。从论文的选题、框架设计到具体内容的修改完善,张老师都倾注了大量心血,给予了我悉心的指导和无私的帮助。张老师严谨的治学态度、深厚的学术素养和诲人不倦的师者风范,令我受益匪浅,并将激励我在未来的学习和工作中不断前进。

感谢在学习期间所有为我授业解惑的老师们,他们的精彩课程为我打下了坚实的专业基础,拓宽了我的学术视野。

感谢实验室的各位同门和同学们,在论文写作过程中,我们相互探讨、共同进步,你们的陪伴和支持让我在科研道路上不再孤单。特别感谢XXX同学在数据处理方面给予的宝贵帮助,以及XXX同学在论文修改过程中提出的建设性意见。

感谢参与本论文评审和提出宝贵意见的各位专家学者,他们的真知灼见为论文的完善提供了重要指导。

最后,我要向我的家人表示最深切的感谢。你们的理解、支持和鼓励是我能够安心完成学业和论文写作的坚强后盾。

由于本人学识水平有限,论文中难免存在疏漏和不足之处,恳请各位老师和专家批评指正。

\bibliographystyle{plain} % 指定参考文献样式(如plain、IEEEtran、apalike等)
\bibliography{resource} % 引用.bib文件(无需写后缀)


\end{document}